\documentclass[12pt,twoside,a4paper,parskip]{scrbook}
\usepackage[utf8]{inputenc}
\usepackage{csquotes}
\usepackage[ngerman]{babel}
\usepackage{floatflt}
\usepackage{subfigure}
\usepackage[pdftex]{graphicx}
\usepackage[hidelinks]{hyperref}
\usepackage{color}
\usepackage{amssymb}
\usepackage{textcomp}
\usepackage{nicefrac}
\usepackage{scrhack}
\usepackage{pdfpages}
\usepackage{float}
\usepackage{pdflscape}
\usepackage{subfigure}
\usepackage{pdfpages}
\usepackage[verbose]{placeins}
\usepackage[nouppercase,headsepline,plainfootsepline]{scrlayer-scrpage}
\usepackage{listings}
\usepackage{xcolor}
\usepackage{color}
\usepackage{caption}
\usepackage{subfigure}
\usepackage{epstopdf}
\usepackage{longtable}
\usepackage{setspace}
\usepackage{booktabs}
\usepackage[style=numeric,backend=biber]{biblatex}
\bibliography{literatur}


%%%%%%%%%%%%%%%%%%%
%% definitions
%%%%%%%%%%%%%%%%%%%
\def\BaAuthor{Lennard Rose, Jochen Schmidt, Moritz Zeitler}
\def\BaAuthorStudyProgram{Informatik} %% Wirtschaftsinformatik, E-Commerce, Informationssysteme
\def\BaType{Projektarbeit} %% Masterarbeit
\def\BaTitle{Entwicklung einer Data Mining Plattform f\"ur Corona Daten}
\def\BaSupervisorOne{Prof. Rott}
\def\BaSupervisorTwo{Prof. Fertig}
\def\BaDeadline{\today}

\ifdefined\iswithfullname
  \def\ShowBaAuthor{\BaAuthor}
\else
  \def\ShowBaAuthor{N.~N.}
\fi

\hypersetup{
pdfauthor={\ShowBaAuthor},
pdftitle={\BaTitle},
pdfsubject={Subject},
pdfkeywords={Keywords}
}

%%%%%%%%%%%%%%%%%%%
%% configs to include
%%%%%%%%%%%%%%%%%%%
\colorlet{punct}{red!60!black}
\definecolor{background}{HTML}{EEEEEE}
\definecolor{delim}{RGB}{20,105,176}
\colorlet{numb}{magenta!60!black}

\definecolor{gray}{rgb}{0.4,0.4,0.4}
\definecolor{darkblue}{rgb}{0.0,0.0,0.6}
\definecolor{cyan}{rgb}{0.0,0.6,0.6}

\definecolor{pblue}{rgb}{0.13,0.13,1}
\definecolor{pgreen}{rgb}{0,0.5,0}
\definecolor{pred}{rgb}{0.9,0,0}
\definecolor{pgrey}{rgb}{0.46,0.45,0.48}

\lstset{
  basicstyle=\ttfamily,
  columns=fullflexible,
  showstringspaces=false,
  commentstyle=\color{gray}\upshape
  linewidth=\textwidth
}

\lstdefinelanguage{json}{
    basicstyle=\normalfont\ttfamily,
    numbers=left,
    numberstyle=\scriptsize,
    stepnumber=1,
    numbersep=8pt,
    showstringspaces=false,
    breaklines=true,
    backgroundcolor=\color{background},
    literate=
     *{0}{{{\color{numb}0}}}{1}
      {1}{{{\color{numb}1}}}{1}
      {2}{{{\color{numb}2}}}{1}
      {3}{{{\color{numb}3}}}{1}
      {4}{{{\color{numb}4}}}{1}
      {5}{{{\color{numb}5}}}{1}
      {6}{{{\color{numb}6}}}{1}
      {7}{{{\color{numb}7}}}{1}
      {8}{{{\color{numb}8}}}{1}
      {9}{{{\color{numb}9}}}{1}
      {:}{{{\color{punct}{:}}}}{1}
      {,}{{{\color{punct}{,}}}}{1}
      {\{}{{{\color{delim}{\{}}}}{1}
      {\}}{{{\color{delim}{\}}}}}{1}
      {[}{{{\color{delim}{[}}}}{1}
      {]}{{{\color{delim}{]}}}}{1},
}

\lstset{language=xml,
  morestring=[b]",
  morestring=[s]{>}{<},
  morecomment=[s]{<?}{?>},
  stringstyle=\color{black},
  numbers=left,
  numberstyle=\scriptsize,
  stepnumber=1,
  numbersep=8pt,
  identifierstyle=\color{darkblue},
  keywordstyle=\color{cyan},
  backgroundcolor=\color{background},
  morekeywords={xmlns,version,type}% list your attributes here
}

\lstset{language=Java,
  showspaces=false,
  showtabs=false,
  tabsize=4,
  breaklines=true,
  keepspaces=true,
  numbers=left,
  numberstyle=\scriptsize,
  stepnumber=1,
  numbersep=8pt,
  showstringspaces=false,
  breakatwhitespace=true,
  commentstyle=\color{pgreen},
  keywordstyle=\color{pblue},
  stringstyle=\color{pred},
  basicstyle=\ttfamily,
  backgroundcolor=\color{background},
%  moredelim=[il][\textcolor{pgrey}]{$$},
%  moredelim=[is][\textcolor{pgrey}]{\%\%}{\%\%}
}

\newcommand*{\forcetwosidetitle}[1][1]{%
 \begingroup
   \cleardoubleoddpage
   \KOMAoptions{titlepage=true}% useful e.g. for scrartcl
   \csname @twosidetrue\endcsname
   \maketitle[{#1}]
 \endgroup
}


\begin{document}


%%%%%%%%%%%%%%%%%%%
%% Titelseite
%%%%%%%%%%%%%%%%%%%


\frontmatter
\titlehead{%  {\centering Seitenkopf}
  {Hochschule für angewandte Wissenschaften Würzburg-Schweinfurt\\
   Fakultät Informatik und Wirtschaftsinformatik}}
\subject{\BaType}
\title{\BaTitle\\[15mm]}
\subtitle{\normalsize{vorgelegt an der Hochschule f\"{u}r angewandte Wissenschaften W\"{u}rzburg-Schweinfurt in der Fakult\"{a}t Informatik und Wirtschaftsinformatik zum Abschluss eines Studiums im Studiengang \BaAuthorStudyProgram}}
\author{\ShowBaAuthor}
\date{\normalsize{Eingereicht am: \BaDeadline}}
\publishers{\
  \normalsize{Erstpr\"{u}fer: \BaSupervisorOne}\\
  \normalsize{Zweitpr\"{u}fer: \BaSupervisorTwo}\\
}
\forcetwosidetitle


%%%%%%%%%%%%%%%%%%%
%% abstract
%%%%%%%%%%%%%%%%%%%

\section*{Zusammenfassung}

TODO

\section*{Abstract}

TODO

\newpage
\chapter*{Danksagung}



%%%%%%%%%%%%%%%%%%%
%% Inhaltsverzeichnis
%%%%%%%%%%%%%%%%%%%
\tableofcontents



%%%%%%%%%%%%%%%%%%%
%% Main part of the thesis
%%%%%%%%%%%%%%%%%%%
\mainmatter

\chapter{Einführung}\label{ch:intro}

\chapter{Datenablagekonzept}
F\"ur die Datenablage wurde zun\"achst ein entsprechendes Konzept entwickelt. Dieses Konzept basiert auf den Daten die abgelegt werden sollen. Hierzu wird zun\"achst ein \"Uberblick \"uber die zu ablegenden Daten gegeben:
\begin{itemize}
	\item Corona Nachrichten/Artikel
	\item Corona Maßnahmen
	\item Wetterdaten
	\item 'Querdenker' Telegram Gruppen
\end{itemize}
Jede einzelne Datenquelle wird nun genauer beleuchtet und ein entsprechendes Datenablagekonzept erarbeitet. Da die Konzepte sich aber sehr \"ahneln wird der Hauptteil der Erkl\"arung bei der gr\"ossten Komponente den Corona-Artikeln zu finden sein.
\section{Corona Nachrichten/Artikel}
Die Corona Nachrichten bzw. Artikel werden in zwei unterschiedlichen Technologien abgelegt. Zun\"achst betrachten wir die Rohdaten. Wie bereits erkl\"art wird jeder Artikel in Rohformat gespeichert. Diese Daten sind aber f\"ur die Auswertung und zur \"Ubersicht der Datensammlung erst mal unwichtig. Dies bedeutet das diese Daten nicht in einer Datenbank indiziert, sondern nur im HDFS abgelegt werden. Die Geschwindigkeit des HDFS f\"ur den Datenzugriff ist ausreichend um die Daten bei einer genauen Auswertung ad-hoc zu lesen. Streng genommen k\"onnte man sogar argumentieren dass das HDFS nichts anderes als ein 'key-value store' ist. So wird im Hintergrund eine Datei die unter einem gewissen Pfad abgelegt worden ist f\"ur den User als klassischer Dateipfad angezeigt (Ordner durch 'forward-slashes' getrennt und am Ende des Pfades ein Dateiname), intern aber der Pfad einen key darstellt. Dies ist n\"otig um die Verteilung und Ausfallsicherheit der Daten \"uber mehrere Knoten gew\"ahrleisten zu k\"onnen. Allerdings ist dies f\"ur den User faktisch nicht bemerkbar, da dieser immer mit den entsprechenden Pfaden arbeitet.
Um nun den Kreis zu schliessen werden nicht nur Meta-Daten der Artikel gespeichert sondern auch der komplette Artikel im Rohformat um Datenverlust vorzubeugen und die Konsistenz der Daten sp\"ater noch pr\"ufen zu k\"onnen. Die Rohdaten werden dann abgelegt unter dem Pfad: '/datakraken/articles/\$bundesland\$/\$Zeitung\$/\$Datum\$/\$ArtikelId\$\_\$TimeStamp\$'. Dieser Pfad wird dann zu den Meta-Daten hinzugef\"ugt.
Nun zu den Meta-Daten, diese kommen werden vom entsprechenden Scraper erzeugt. Dies Daten werden dann im Elasticsearch Cluster abgelegt unter entsprechendem Index Namen. Dies hilft dabei eine \"Ubersicht zu den Daten zu erhalte, einen Status zu bekommen in welchem Mass die Daten in das System kommen und erm\"oglicht eine rudiment\"are Analyse. Die Daten an sich sind definiert durch die entsprechende Config
\section{Corona Maßnahmen}
Das soeben erw\"ahnte Konzept der Nachrichten wird genauso f\"ur die Massnahmen verwendet. Rohdaten werden im HDFS abgelegt, w\"ahrend die beschreibenden Daten im Elasticsearch indiziert werden. Diese Massnahmen werden mit folgendem Pattern abgelegt: '/datakraken/measures/\$bundesland\$/\$Zeitung\$/\$Datum\$/\$ArtikelId\$\_\$TimeStamp\$'
\section{Wetterdaten}
Die Wetterdaten hingegen werden ausschließlich im Elasticsearch abgelegt. Das von der API gelieferte Format ist bereits im JSON Format und kann dadurch direkt, und viel wichtiger komplett indiziert werden.
\section{'Querdenker' Telegram Gruppen}
Im Bereich der Telegram Gruppen muss hier differenziert werden. Die verschickten Nachrichten an sich werden mit den Meta-Daten zusammen im Elasticsearch abgelegt. H\"angt allerdings an der Nachricht noch eine Datei ein, oder ein Link, wird die Datei im HDFS abgelegt. Dazu wird eine Referenz erstellt und mit im Elasticsearch abgelegt. Die Ablage Struktur im HDFS ist in folgendem Pattern: '/datakraken/telegram/\$GruppenName\$/\$Datum\$/\$NachrichtId\$\_\$TimeStamp\$'
\chapter{Problemstellung}

\chapter{Lösung}

\chapter{Evaluierung}

\chapter{Zusammenfassung}


\backmatter
%%%%%%%%%%%%%%%%%%%
%% create figure list
%%%%%%%%%%%%%%%%%%%

\listoffigures
\addcontentsline{toc}{chapter}{Verzeichnisse}

%%%%%%%%%%%%%%%%%%%
%% create tables list
%%%%%%%%%%%%%%%%%%%
\listoftables

%%%%%%%%%%%%%%%%%%%
%% create listings list
%%%%%%%%%%%%%%%%%%%
%\lstlistoflistings
%\addcontentsline{toc}{chapter}{Listings}

\cleardoublepage
\phantomsection
\addcontentsline{toc}{chapter}{Literatur}
\printbibliography

%%%%%%%%%%%%%%%%%%%
%% declaration on oath
%%%%%%%%%%%%%%%%%%%

\addchap{Eidesstattliche Erklärung}

Hiermit versichere ich, dass ich die vorgelegte Bachelorarbeit selbstständig verfasst und noch nicht anderweitig zu Prüfungszwecken vorgelegt habe. Alle benutzten Quellen und Hilfsmittel sind angegeben, wörtliche und sinngemäße Zitate wurden als solche gekennzeichnet.

\vspace{20pt}
\begin{flushright}
$\overline{~~~~~~~~~~~~~~~~~\mbox{\ShowBaAuthor, am \today}~~~~~~~~~~~~~~~~~}$
\end{flushright}

\addchap{Zustimmung zur Plagiatsüberprüfung}

Hiermit willige ich ein, dass zum Zwecke der Überprüfung auf Plagiate meine vorgelegte Arbeit in digitaler Form an PlagScan (www.plagscan.com) übermittelt und diese vorübergehend (max. 5~Jahre) in der von PlagScan geführten Datenbank gespeichert wird sowie persönliche Daten, die Teil dieser Arbeit sind, dort hinterlegt werden.

\begin{small}
Die Einwilligung ist freiwillig. Ohne diese Einwilligung kann unter Entfernung aller persönlichen Angaben und Wahrung der urheberrechtlichen Vorgaben die Plagiatsüberprüfung nicht verhindert werden. Die Einwilligung zur Speicherung und Verwendung der persönlichen Daten kann jederzeit durch Erklärung gegenüber der Fakultät widerrufen werden.
\end{small}

\vspace{20pt}
\begin{flushright}
$\overline{~~~~~~~~~~~~~~~~~\mbox{\ShowBaAuthor, am \today}~~~~~~~~~~~~~~~~~}$
\end{flushright}

\end{document}
